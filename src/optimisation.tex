% 
% Методы оптимизации
%

\begin{enumerate}
\item Общая  постановка  экстремальной  задачи  в  произвольном  нормированном  пространстве.  Различные  виды  ограничений.  Допустимые  точки.  Понятие  решения  задачи.  Определения  локального  и  глобального  экстремумов.

\item Гладкая  задача  без  ограничений  в  конечномерном   пространстве.  Теорема  о  необходимых  условиях  экстремума  первого  порядка (теорема  Ферма).  Условия  второго  порядка.  Матрица  вторых  смешанных  производных  (матрица  Гессе).  Критерии  положительной  и  отрицательной  определённости  симметрической  матрицы. (критерий  Сильвестра).  Теорема  о  достаточных  условиях  экстремума (максимума  и  минимума)  в  конечномерной  задаче  без  ограничений.

\item Гладкая  задача  с  ограничениями  в  форме  равенств  в  конечномерном  пространстве.  Функция  Лагранжа.  Теорема  о  необходимых  условиях  экстремума  в  задаче  с ограничениями  в  форме  равенств.  Условие  регулярности  и  его  теоретическое  значение.  Общий  метод  решения  задачи  с равенствами.  Алгоритмический  смысл  необходимых  условий  экстремума.  Теорема  о  достаточных  условиях  экстремума  в гладкой  задаче  с  равенствами.

\item Гладкая  задача  с  ограничениями  в  форме  равенств  и  неравенств  в  конечномерном  пространстве.  Функция  Лагранжа.  Теорема  о  необходимых  условиях  экстремума  в  задаче  с  ограничениями  в  форме  равенств  и  неравенств.  Особенности  необходимых  условий  в  задачах  с ограничениями  в  форме  равенств   и  неравенств.  

\item Классическое  вариационное  исчисление. Задача  Больца  без  ограничений.  Теорема  о необходимых  условиях  экстремума.  Алгоритмический  смысл  необходимых  условий  экстремума  в  задаче  Больца.  Использование  метода  непосредственной  проверки  для  доказательства  оптимальности  допустимой  экстремали.

\item Классическое  вариационное  исчисление.  Простейшая  задача.  Теорема  о необходимых  условиях  экстремума  (условия  первого  порядка).  Условия  второго порядка  в  простейшей  задаче  (условия  Лежандра  и  Якоби  для  одномерного  варианта).  Теоремы  о  необходимых  условиях  второго  порядка  и  достаточных  условиях  экстремума  в  простейшей  задаче.

\item Классическое  вариационное  исчисление.  Задача  с  общими  граничными  условиями.  Теорема о  необходимых  условиях  экстремума.  Возможность  решения  системы,  состоящей  из  необходимых  условий  и  ограничений.  Особенности  условий  трансверсальности  в  задаче  с  общими  граничными  условиями.  Использование  метода  непосредственной  проверки  для  доказательства  оптимальности  допустимой  экстремали.

\end{enumerate}



