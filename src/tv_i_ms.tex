%
% Теория вероятностей и математическая статистика
%

\begin{enumerate}

\item Случайные события и действия над ними. Определение и свойства вероятности. Формула  сложения. Классическое определение вероятности. Условные вероятности. Независимость событий. Формула полной вероятности. Формула Байеса.


\item Случайные величины. Функция распределения случайной величины и ее свойства.  Дискретные случайные величины. Примеры распределений дискретных случайных величин (биномиальное, геометрическое,  пуассоновское).  Абсолютно непрерывные случайные  величины. Плотность распределения и ее свойства. Примеры абсолютно непрерывных распределений (равномерное, нормальное, показательное). 

\item Числовые характеристики случайных величин. Свойства математического ожидания, дисперсии, ковариации и корреляции. Некоррелированность и независимость случайных величин.
\item Неравенство Чебышева. Закон больших чисел, примеры его применений.
Центральная предельная теорема и теорема Муавра - Лапласа, как ее частный случай.

\item Понятие выборки.   Оценки функции распределения и плотности: эмпирическая функция распределения и гистограмма.

\item Точечные оценки и их свойства. Метод моментов. Метод максимального правдоподобия. Примеры (среднее нормального распределения при известной дисперсии).

\item Построение доверительного интервала. Пример (доверительный интервал для среднего значения нормального распределения при известной дисперсии).

\item Проверка гипотез. Критерий, ошибки первого и второго рода, построение критической области. Проверка гипотезы о виде распределения по критерию хи-квадрат.

\end{enumerate}




