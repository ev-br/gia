% 
% Теория функций комплексного переменного
%

\begin{enumerate}

\item Необходимые и достаточные условия дифференцируемости функции комплексного переменного во внутренней точке области (доказательство теоремы). Гармонические и сопряженные гармонические функции (определения). Восстановление в односвязной области  аналитической функции по ее действительной или мнимой части (вывод формулы). 

\item Геометрический смысл  аргумента и модуля  производной функции комплексного переменного.
Конформное отображение (определение). Круговое свойство дробно-линейного отображения (доказательство)

\item Теорема Коши для односвязной области и ее обобщение на случай многосвязной области (формулировки). Вывод интегральной формулы Коши. Определение интеграла типа Коши. Существование производных всех порядков для функции, аналитической в области (доказательство теоремы).

\item Теорема Тейлора (формулировка). Теорема единственности определения аналитической функции (формулировка) и ее следствия. Теорема Лорана (доказательство). Классификация изолированных особых точек однозначной аналитической функции. Теорема Сохоцкого-Вейерштрасса (формулировка).

\item Определение и формулы вычисления вычета  аналитической функции в изолированной особой точке. Основная теорема теории вычетов (доказательство). Лемма Жордана (доказательство).

\end{enumerate}
