%
% Линейная алгебра и аналитическая геометрия.
%

\begin{enumerate}

\item Скалярное, векторное и смешанное произведения в $\mathbb{R}^3$. Вычисление произведений в координатах. Объем параллелепипеда и тетраэдра. Площадь треугольника в пространстве.

\item Общее уравнение плоскости и канонические уравнения прямой в пространстве. Условия параллельности и перпендикулярности  прямой и плоскости.  Вычисление расстояния от точки до плоскости.

\item Кривые второго порядка. Эллипс, гипербола и парабола как геометрические места точек. Канонические уравнения кривых второго порядка и их графики.

\item Элементарные преобразования матриц. Приведение к ступенчатому и главному ступенчатому виду. Решение линейных систем методом Гаусса. Невырожденная матрица, ранг матрицы. Теорема Кронекера-Капелли.

\item Умножение матриц.  Свойства операции умножения. Обратная матрица и критерий ее существования. Определитель. Определитель и невырожденность. Определитель  произведение матриц.

\item Понятие линейного пространства. Пространство $\mathbb{R}^n$. Линейная зависимость и независимость векторов. Базис.  Размерность пространства. Линейные подпространства.

\item Линейные операторы и их матрицы.  Собственные значения и собственные векторы линейного оператора. Характеристический многочлен. Линейная независимость системы собственных векторов, имеющих различные собственные значения.

\end{enumerate}





