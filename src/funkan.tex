%
% Функциональный анализ
%

\begin{enumerate}
\item Дайте определение сжимающего отображения. Сформулируйте теорему о неподвижной точке сжимающего отображения. 
Пользуясь теоремой о неподвижной точке сжимающего отображения, докажите, что уравнение $x(t) =\displaystyle{\int_0^t \frac{ds}{2+x^2(s)}}$ имеет единственное решение в пространстве $C[0,1]$. Укажите алгоритм поиска решения. Вычислите первые два приближения к решению, взяв исходную функцию $x_0(t) \equiv 0$. Оцените точность найденного приближенного решения. 

\item Дайте определение регулярного значения оператора, резольвентного множества оператора, спектра оператора. 
Вычислите норму и спектр оператора левого сдвига в $l_2$:  $$A: (x_1,\,x_2,\ldots)\mapsto(x_2,\,x_3,\ldots).$$
\end{enumerate}

\bigskip
[1]  Колмогоров А.Н., Фомин С.В. Элементы теории функций и функционального анализа. – М.: Наука, 1976 (а также более поздние издания).




