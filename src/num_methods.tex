% 
% Численные методы
%

\begin{enumerate}
\item Метод простой итерации для решения систем линейных алгебраических уравнений. Достаточные условия сходимости сходимость метода простой итерации. Априорные и апостериорные оценки погрешности. Необходимые и достаточные условия сходимости (без доказательства).

\item Метод Ньютона для решения нелинейных уравнений: расчетные формулы, геометрический смысл, связь  с методом простой итерации. Скорость сходимости метода простой итерации  и метода Ньютона. Модификации метода Ньютона для кратных корней.

\item Норма вектора и матрицы: аксиомы нормы, примеры векторных и матричных норм. Число обусловленности матрицы. Обусловленность задачи решения системы линейных алгебраических уравнений (без доказательства).

\item $LU$-разложение невырожденной матрицы. Вычислительная сложность построения $LU$-разложения. Решение систем линейный алгебраических уравнений с помощью $LU$-разложения. Выбор главного элемента: стратегии выбора, вычислительная сложность.  

%\item Постановка задачи интерполяции. Интерполяция многочленами: существование и единственность интерполяционного многочлена. Интерполяционный многочлен в форме Лагранжа. Погрешность интерполяции, явление Рунге.  

\item Явный и неявный метод Эйлера решения задачи Коши обыкновенного дифференциального уравнения (ОДУ) первого порядка. Порядок аппроксимации метода Эйлера. Понятие абсолютной и относительной устойчивость конечно-разностной схемы решения ОДУ. Исследование устойчивости явного и неявного методов Эйлера для уравнения $du/dt = \lambda u$.

\end{enumerate}
