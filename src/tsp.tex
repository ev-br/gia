%
% Случайные процессы и теория массового обслуживания 
%


\begin{enumerate}

\item Дать определение простого процесса восстановления и процесса восстановления
с запаздыванием. Дать определение функции восстановления. Выписать
функцию восстановления для простого процесса восстановления, у которого
интервалы между восстановлениями имеют экспоненциальное распределение.
Выписать интегральные уравнения восстановления для простого процесса
восстановления и процесса восстановления с запаздыванием. Сформулировать
элементарную и узловую теоремы восстановления.

\item Марковские цепи. Определение. Однородные марковские цепи. Дать
определение матрицы переходных вероятностей за один и за несколько
шагов. Выписать формулу для вычисления вероятностей перехода за $n$
шагов. Вывести формулу конечномерного распределения для однородной
марковской цепи, используя марковское свойство и свойство однородности
цепи.

\item Марковские цепи. Классификация состояний и цепей. Дать определения
возвратности, существенных и несущественных состояний, положительных
и нулевых состояний. Сформулировать связь указанных выше свойств состояний
марковской цепи. Дать определение периода состояния. Альтернатива
солидарности.

\item Дать определение марковского процесса с непрерывным временем и
дискретным множеством состояний.  Дать определение переходных вероятностей
и сформулировать их свойства. Дать определие однородного марковского
процесса. Дать определение интенсивностей перехода, выписать уравнения
Колмогорова. 

\item Дать определение марковского процесса гибели и размножения с конечным
и счетным множеством состояний. Выписать интенсивности перехода, уравнения
Колмогорова, провести анализ вероятностей состояний при бесконечном
времени. 

\item Дать определение процесса Пуассона. Сформулировать его основные
свойства и характеристики. Вычислить вероятность $P\left\{ X\left(2\right)\geq2|\, X\left(1\right)=1\right\} $,
где $X\left(t\right)$ - процесс Пуассона с параметром $\lambda$,
$X\left(0\right)=0$.

\end{enumerate}

