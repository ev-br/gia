%
% Математический анализ
%

\begin{enumerate}
\item Докажите теорему Коши о промежуточном значении непрерывной функции на отрезке. Изложите метод решения уравнений $f(x) = 0$ методом деления отрезка пополам. Докажите, что уравнение $\cos{x} = x$ имеет решение на отрезке $[0, 1]$. Как его найти с точностью 0.001?

\item Выведите формулу Тейлора-Лагранжа для функций одного переменного. Вычислите число $e$ с точностью 0.01. 

\item Определите понятия непрерывной и дифференцируемой функции одного переменного. Докажите теорему о дифференцируемости интеграла с переменным верхним пределом. Выведите формулу Ньютона-Лейбница.

\item Расскажите о методе подстановки в определенном интеграле. С его помощью вычислите 
$$
\int_0^a\!\sqrt{a^2 - x^2}\, dx\;, \qquad \int_1^2\! \frac{x\,dx}{x^4 + 1}\;.
$$
Запишите формулу интегрирования по частям для определенного интеграла. Вычислите 
$$
\int_1^e\! x\, \ln^2{x} \, dx
$$

\item Дайте определения сходящегося числового ряда и несобственного интеграла на $[1, +\infty]$. Сформулируйте интегральный признак сходимости числового ряда. Исследуйте на сходимость при разных значениях $\alpha > 0$ интегралы Дирихле и ряды Дирихле
$$
\int_1^{+\infty}\! \frac{dx}{x^\alpha} \;,\qquad\qquad
%
\sum_{n=1}^{\infty} \frac{1}{n^\alpha} \;.
$$

\item Дайте определение точки локального экстремума функции нескольких переменных. Выведите необходимое условие локального экстремума для дифференцируемых функций. Для функций двух и трех переменных сформулируйте достаточное условие локального экстремума и его отсутствия. Найдите точки локального экстремума функции $z = 2x^2 y + y^3 - 3y$ и укажите, к какому типу они относятся.

\item Разложите функцию $\sgn{x}$ в ряд Фурье на отрезке $[-\pi, \pi]$. Записав равенство Парсеваля для этой функции получите формулу
$$
1 + \frac{1}{3^2} + \frac{1}{5^2} + \frac{1}{7^2} + \cdots = \frac{\pi^2}{8} \;.
$$

\item Дайте определение криволинейного интеграла векторного поля в $\mathbb{R}^2$ и объясните его физический смысл. Что такое потенциальное векторное поле? Установите потенциальность векторного поля $(\sqrt{x} + e^y, y\sqrt{y} + x e^y)$ и найдите его работу на пути от точки $A(1, 1)$ до точки $B(2, 4)$.


\item Дайте определение поверхностных интегралов от функции и от векторного поля в $\mathbb{R}^3$. Поясните их физический смысл. Найдите поток поля $(x + y, y+z, z+x)$ через границу шара $x^2 + y^2 + z^2 \leqslant 1$ в направлении внешней нормали.


\end{enumerate}



