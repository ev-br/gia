%
% Дифференциальные уравнения
%

\begin{enumerate}[resume]
\item Дайте определение задачи Коши обыкновенного дифференциального уравнения первого порядка, разрешённого относительно производной. Сформулируйте теорему Коши-Липшица. Из каких условий на функцию $f(x, y)$ следует выполнение условия Липшица (с обоснованием)?  Покажите, что условие Липшица слабее этих рассмотренных условий. Приведите пример нарушения единственности решения рассматриваемой задачи Коши, если $\dfrac{\partial f(x,\,y)}{\partial y}$ не существует. 

\item Сформулируйте теорему об общем решении линейного неоднородного уравнения с постоянными коэффициентами с правой частью в виде квазимногочлена. Докажите её для линейного однородного уравнения в случае некратных корней характеристического многочлена.
Решите уравнение
 $$y''-2y'=8 \sin 2 x+4x.$$

\item Дайте определение фундаментальной системы решений линейной однородной системы дифференциальных уравнений (ЛОС). Докажите, что решения ЛОС $n$-го порядка образуют линейное пространство размерности $n$. 

Решите данную систему, укажите какую-нибудь её фундаментальную систему решений:


\begin{equation*}
 \begin{cases}
   \dot{x}=2 x-y+z\\
\dot{y}=x+2y-z\\
\dot{z}=x-y+2z.
  \
 \end{cases}  
\end{equation*}
$(\lambda_{1}=1, \lambda_2=2, \lambda_3=3)$

\item Изложите классификацию изолированных особых точек уравнения $\dfrac{dy}{dx}=\dfrac{cx+dy}{ax+by}$ в терминах характеристических корней. Приведите примеры уравнений, для которых $(0, 0)$ является узелом, седлом, центром, фокусом.

\end{enumerate}

\bigskip
[1] Филиппов А.Ф. Введение в теорию дифференциальных уравнений. – М. Изд-во «УРСС». - 2004.

[2] Петровский И.Г. Лекции по теории обыкновенных дифференциальных уравнений. – М. Изд-во «УРСС». - 2003.


